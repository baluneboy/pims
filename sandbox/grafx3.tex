\documentclass[journal,comsoc]{IEEEtran}

\usepackage[
  letterpaper,
  landscape,
  top=0.5in,
  bottom=0.5in,
  left=0.5in,
  right=0.3in,
  headheight=17pt, % as per the warning by fancyhdr
  includehead,includefoot,
  heightrounded, % to avoid spurious underfull messages
]{geometry}

\usepackage{graphicx}
\usepackage{lipsum}
\usepackage{fancyhdr}
\usepackage{lastpage}
\usepackage{fix-cm}
\usepackage[per-mode=symbol]{siunitx}
\usepackage{textcase}
\usepackage[tablename=Table]{caption}

\usepackage{mathptmx}
\usepackage{bm}
\usepackage{amsmath}

\usepackage[shortlabels]{enumitem}

\usepackage{hyperref}
\hypersetup{
    colorlinks=true,
    linkcolor=blue,
    filecolor=magenta,      
    urlcolor=cyan,
}
\urlstyle{same}

\captionsetup[table]{
    labelsep=period,
    justification=centering,
    position=bottom,
}

\fancyhf{}
\pagestyle{fancy}

% Header
\fancyhead[L]{\fontsize{8}{6} \selectfont \textsc{Vehicle}}
\fancyhead[C]{\textbf{\fontsize{12}{10} \selectfont Russian Segment (RS) Optimized Propellant Maneuver (OPM) to -XVV on GMT 2019-12-28}}
\fancyhead[R]{\textsc{\fontsize{8}{6} \selectfont page \thepage/\pageref{LastPage}}}

% Footer
\fancyfoot[L]{\fontsize{8}{6} \selectfont \textsc{Quasi-Steady}}
\fancyfoot[C]{\fontsize{6}{4} \selectfont modified \today}
\fancyfoot[R]{\fontsize{8}{6} \selectfont \textsc{Placeholder}}

\begin{document}

\renewcommand{\thetable}{\arabic{table}}
\renewcommand \thesection{\arabic{section}} % change from Roman numerals to Arabic

%%%%%%%%%%%%%%%%%%%%%%%%%%%%%%%%%%%%%%%%%%%%%%%%%%%%%%%%%%%%%%%%%%%%%%%%%%%%%%%%%%%%%%%%%%%%%%%%%
% Intro
\section{Introduction}

\lipsum[2]

\begin{figure}[thb]
	\centering  
	\includegraphics[width=\columnwidth]{example-image}
	\caption{Perhaps a photo or graphic of a visiting vehicle.}
	\label{fig:graysquare}
\end{figure}

Figure \ref{fig:graysquare} does not really show a photograph of anything worth noting.  \lipsum[1]


%%%%%%%%%%%%%%%%%%%%%%%%%%%%%%%%%%%%%%%%%%%%%%%%%%%%%%%%%%%%%%%%%%%%%%%%%%%%%%%%%%%%%%%%%%%%%%%%%
% Qualify
\section{Qualify}

\lipsum[1]

%\vspace{\baselineskip}

\begin{table}
	\centering
	\begin{tabular}{|c|l|l|}
	\hline
	\textbf{$\text{Pitch Angle } (\alpha)$}&\multicolumn{2}{c|}{$\mathbf{V_w=9.6\,m/s}$}\\
	\hline
	$\SI{-10}{\degree}$&$c_DA=0.0143$&$c_LA=0.0054$\\
	\hline
	$\SI{0}{\degree}$&$c_DA=0.0142$&$c_LA=0.0017$\\
	\hline
	$\SI{10}{\degree}$&$c_DA=0.0119$&$c_LA=-0.0028$\\
	\hline
	$\SI{20}{\degree}$&$c_DA=0.0139$&$c_LA=0.0018$\\
	\hline
	$\SI{30}{\degree}$&$c_DA=0.0119$&$c_LA=0.0033$\\
	\hline
	\end{tabular}
	\caption{Information from ISS as-flown timeline showing wind speed of \SI{9.6}{\meter\per\second} and, while we are discussing this, different pitch angles ($\beta=\SI{0}{\degree}$).}
	\label{table:draglift1}
\end{table}

\lipsum[1]

For further references see \href{http://www.sharelatex.com}{Something Linky} or go to the next url: \url{http://www.sharelatex.com}.

List are really easy to create...
 
\begin{itemize}
  \item One entry in the list.
  \item Another entry in the list.
\end{itemize}

Here is an ordered list:

\begin{enumerate}[(1)] % (1), (2), (3), ...
\item Apple
\item Banana
\end{enumerate}

Here is an ordered list:

\begin{enumerate}[(a)]
  \item The labels consists of sequential numbers.
  \item The numbers starts at 1 with every call to the enumerate environment.
\end{enumerate}

\lipsum[1]

\begin{figure}[thb]
	\centering  
	\includegraphics[width=\columnwidth]{myspectrogram.pdf}
	\caption{Another interesting y-axis feature.}
	\label{fig:specgram}
\end{figure}

Figure \ref{fig:specgram} on page \pageref{fig:specgram} does not really show a photograph of anything worth noting.  \lipsum

\end{document}
